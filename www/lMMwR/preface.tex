%%%%%%%%%%%%%%%%%%%%%% pref.tex %%%%%%%%%%%%%%%%%%%%%%%%%%%%%%%%%%%%%
%
% sample preface
%
% Use this file as a template for your own input.
%
%%%%%%%%%%%%%%%%%%%%%%%% Springer-Verlag %%%%%%%%%%%%%%%%%%%%%%%%%%

\preface

R is a freely available implementation of John Chambers' award-winning
S language for computing with data.  It is ``Open Source'' software
for which the user can, if she wishes, obtain the original source code
and determine exactly how the computations are being performed.  Most
users of R use the precompiled versions that are available for recent
versions of the Microsoft Windows operating system, for Mac OS X, and
for several versions of the Linux operating system.  Directions for
obtaining and installing R are given in Appendix~\ref{app:Obtaining}.

Because it is freely available, R is accessible to anyone who cares to
learn to use it, and thousands have done so. Recently many prominent
social scientists, including John Fox of McMaster University and Gary
King of Harvard University, have become enthusiastic R users and
developers.

Another recent development in the analysis of social sciences data is
the recognition and modeling of multiple levels of variation in
longitudinal and organizational data.  The class of techniques for
modeling such levels of variation has become known as multilevel
modeling or hierarchical linear modeling.

In this book I describe the use of R for examining, managing, 
visualizing, and modeling multilevel data.


%% "sign" the preface
\vspace{1cm}
\begin{flushright}\noindent
Madison, WI, USA,\hfill {\it Douglas Bates}\\
October, 2005\\
\end{flushright}


