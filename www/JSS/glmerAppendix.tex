\documentclass{jss}
%% need no \usepackage{Sweave.sty}
\usepackage{mathtools}
\usepackage{lineno}

\newcommand{\blockmatrix}[3]{%These end of the line comments are neccessary
\begin{minipage}[t][#2][c]{#1}%
\center%
#3%
\end{minipage}%
}%
\newcommand{\fblockmatrix}[3]{%
\fbox{%
\begin{minipage}[t][#2][c]{#1}%
\center%
#3%
\end{minipage}%
}%
}

\usepackage{etoolbox}
\let\bbordermatrix\bordermatrix
\patchcmd{\bbordermatrix}{8.75}{4.75}{}{}
\patchcmd{\bbordermatrix}{\left(}{\left[}{}{}
\patchcmd{\bbordermatrix}{\right)}{\right]}{}{}

\newcommand{\bmb}[1]{{\color{red} \emph{#1}}}
\newcommand{\scw}[1]{{\color{blue} \emph{#1}}}
\usepackage[american]{babel}  %% for texi2dvi ~ bug
\usepackage{bm,amsmath,thumbpdf,amsfonts}
\author{
  Steven C. Walker\\McMaster University \And
  Rune Haubo Bojesen Christensen\\Technical University of Denmark\AND
  Douglas Bates\\University of Wisconsin - Madison \And
  Ben Bolker\\McMaster University \AND
  Martin M\"achler\\ETH Zurich
}
\Plainauthor{Steve Walker, Rune Haubo Bojesen Christensen, Douglas Bates, Martin M\"achler, Ben Bolker}
\title{Fitting generalized linear \bmb{and nonlinear?} mixed-effects models using \pkg{lme4}}
\Plaintitle{Fitting generalized linear mixed models using lme4}
\Shorttitle{GLMMs with lme4}
\Abstract{%
\bmb{abstract goes here}
}
\Keywords{%
  sparse matrix methods,
  linear mixed models,
  penalized least squares,
  Cholesky decomposition}
\Address{
  Steven C. Walker\\
  Department of Mathematics \& Statistics \\
  McMaster University \\
  1280 Main Street W \\
  Hamilton, ON L8S 4K1, Canada \\
  E-mail: \email{scwalker@math.mcmaster.ca}
  \par\bigskip
  Rune Haubo Bojesen Christensen \\
  Technical University of Denmark \\
  Matematiktorvet \\
  Building 324, room 220 \\
  2800 Kgs. Lyngby \\
  E-mail: \email{rhbc@dtu.dk}
  \par\bigskip
  Douglas Bates\\
  Department of Statistics, University of Wisconsin\\
  1300 University Ave.\\
  Madison, WI 53706, U.S.A.\\
  E-mail: \email{bates@stat.wisc.edu}
  \par\bigskip
  Martin M\"achler\\
  Seminar f\"ur Statistik, HG G~16\\
  ETH Zurich\\
  8092 Zurich, Switzerland\\
  E-mail: \email{maechler@stat.math.ethz.ch}\\
  % URL: \url{http://stat.ethz.ch/people/maechler}
  \par\bigskip
  Benjamin M. Bolker\\
  Departments of Mathematics \& Statistics and Biology \\
  McMaster University \\
  1280 Main Street W \\
  Hamilton, ON L8S 4K1, Canada \\
  E-mail: \email{bolker@mcmaster.ca}
}
\newcommand{\Var}{\operatorname{Var}}
\newcommand{\abs}{\operatorname{abs}}
\newcommand{\bLt}{\ensuremath{\bm\Lambda_\theta}}
\newcommand{\mc}[1]{\ensuremath{\mathcal{#1}}}
\newcommand{\trans}{\ensuremath{^\prime}}
\newcommand{\yobs}{\ensuremath{\bm y_{\mathrm{obs}}}}
\newcommand*{\eq}[1]{eqn.~\ref{#1}}% or just {(\ref{#1})}
%



\setkeys{Gin}{width=\textwidth}
\begin{document}
%\SweaveOpts{concordance=TRUE}

\section{Appendix: derivation of PIRLS (nAGQ = 0 version)}

We seek to maximize PDEV the unscaled conditional log-likelihood 
for a GLMM over the conditional modes, $\bm u$ and fixed effect
coefficients, $\bm\beta$. This problem is very similar to maximizing
the log-likelihood for a GLM, about which there is a large amount of
work. The standard algorithm for dealing with this kind of problem is
iteratively reweighted least squares (IRLS). Here we modify IRLS by
incorporating a penalty term that accounts for variation in the random
effects, which we call penalized iteratively reweighted least squares (PIRLS). 

The unscaled conditional log-likelihood takes the following form,
\begin{equation}
L(\bm\beta, \bm\theta | \bm y, \bm u) = 
\bm\psi^\top \bm A \bm y - 
\bm a^\top \bm \phi -
\frac{1}{2}\bm u^\top \bm u
\end{equation}
where $\bm\psi$ is the $n$-by-$1$ natural parameter of an exponential family,
$\bm\phi$ is the $n$-by-$1$ vector of cumulant functions, and $\bm A$ is an $n$-by-$n$ diagonal
matrix of prior weights, $\bm a$, which could depend on a dispersion
parameter although we ignore this possibility for now.

The natural parameter, $\bm\psi$, and cumulant function, $\bm\phi$,
depend on a linear predictor,
\begin{equation}
\bm\eta = \bm o + \bm X \bm\beta + \bm Z \bm\Lambda_\theta \bm u
\end{equation}
where $\bm o$ is an \emph{a priori} offset. The specific form of this
dependency is specified by the choice of the exponential family
(e.g. binomial). Furthermore, the mean, $\bm\mu$, of this distribution
is a function of $\bm\eta$, where this function is standardly referred
to as the inverse link function.

Instead of maximizing $L(\bm\beta, \bm\theta | \bm y, \bm u)$, we
typically think of minimizing the penalized
deviance,
\begin{equation}
\mathrm{PDEV} = -2 L(\bm\beta, \bm\theta | \bm y, \bm u)
\end{equation}
To minimize PDEV, we use a penalized iteratively reweighted least
squares (PIRLS) algorithm. There are five steps to deriving PIRLS: 
\begin{enumerate}
\item \textbf{Stationary points of PDEV}
  \begin{itemize}
    \item Find the stationary points of the PDEV criterion.
    \item This step follows McCullagh and Nelder (1989) very closely,
      with only a slight modification for working with conditional
      log-likelihoods.
  \end{itemize}
\item \textbf{Stationary points of PWRSS}
  \begin{itemize}
    \item Find the stationary points of the PWRSS criterion.
    \item These first two steps show that the PDEV and PWRSS criteria
      have the same stationary points, \emph{if} the chosen weights
      used in PWRSS correspond to the prior weights divided by the
      variance function at the minimum of PDEV.
    \item Therefore, if we minimize PWRSS with a correct guess at
      the variance function, then we also minimize PDEV.
    \item However, because this correct variance function is not known \emph{a priori},
      we iteratively solve PWRSS, updating the variance
      function at each iteration.
  \end{itemize}
\item \textbf{Pseudo-data and weighted residuals}
  \begin{itemize}
    \item Express the PWRSS as a simple residual sum of squares using
      pseudo-data and weighted residuals.
    \item This expresses the problem as a standard non-linear least-squares
      problem, which can be solved iteratively using the Gauss-Newton
      method~\citep[\S2.2.3]{bateswatts88:_nonlin}.
  \end{itemize}
\item \textbf{Gauss-Newton}
  \begin{itemize}
    \item Apply the Gauss-Newton method to the non-linear
      least-squares problem.
    \item This yields a sequence of normal equations.
    \item If the solutions of this sequence converge, then they
      converge to the minimizer of PDEV, \textbf{(this is the step I'm
        unsure of)}, beacuse PDEV and PWRSS have the same stationary
      points at convergence.
  \end{itemize}
\item \textbf{Cholesky}
  \begin{itemize}
    \item Solve the normal equations using block Cholesky
      decompositions.
    \item These particular Cholesky decompositions exploit the sparsity
      of some of the blocks of the cross-product matrices in the
      normal equations.
  \end{itemize}
\end{enumerate}

\subsection{Stationary points of PDEV}

Following standard GLM theory (e.g. McCullagh and Nelder 1989), we use the chain rule,
\begin{displaymath}
\frac{d PDEV}{d \bm\beta} = 
\frac{d PDEV}{d \bm\psi}
\frac{d \bm\psi}{d \bm\mu}
\frac{d \bm\mu}{d \bm\eta}
\frac{d \bm\eta}{d \bm\beta}
\end{displaymath}
The first derivative in this chain follow from basic results in GLM theory,
\begin{displaymath}
\frac{d PDEV}{d \bm\psi} = 
-2(\bm y - \bm\mu)^\top \bm A
\end{displaymath}
Again from standard GLM theory, the next two derivatives define the inverse diagonal variance
matrix,
\begin{displaymath}
\frac{d \bm\psi}{d \bm\mu} = \bm V^{-1}
\end{displaymath}
and the diagonal Jaccobian matrix,
\begin{displaymath}
\frac{d \bm\mu}{d \bm\eta} = \bm M
\end{displaymath}
Finally, because $\bm\beta$ affects $\bm\eta$ only linearly,
\begin{displaymath}
\frac{d \bm\eta}{d \bm\beta} = \bm X
\end{displaymath}
Therefore, we have,
\begin{equation}
\frac{d PDEV}{d \bm\beta} = 
-2(\bm y - \bm\mu)^\top \bm A
\bm V^{-1}
\bm M
\bm X
\label{eq:dPDEVdbeta}
\end{equation}
Similarly, differentiating with respect to $\bm u$ we have,
\begin{equation}
\frac{d PDEV}{d \bm u} = 
-2(\bm y - \bm\mu)^\top \bm A
\bm V^{-1}
\bm M
\bm Z \bm\Lambda_\theta +
2\bm u^\top
\label{eq:dPDEVdu}
\end{equation}
Again as in McCullagh and Nelder (1989), instead of trying to find the
roots of these equations directly, we use iterative least-squares
methods that converge on these roots. In particular, we define a
penalized weighted residual sum-of-squares criterion (PWRSS) with the same
roots as PDEV.

\subsection{Stationary points of PWRSS}

The penalized weighted residual sum of squares (PWRSS) is given by,
\begin{equation}
\mathrm{PWRSS} = (\bm y-\bm\mu)^\top \bm W (\bm y-\bm\mu) + \bm u^\top
\bm u
\label{eq:PWRSS}
\end{equation}
where the weights matrix $\bm W = \bm A \bm V^{-1}$. If we hold $\bm
W$ fixed at an initial
guess for the variance function, $\bm V = \bm V_0$, and differentiate with
respect to $\bm\beta$ and $\bm u$ we obtain,
\begin{equation}
\frac{d \mathrm{PWRSS}}{d \bm\beta} = 
-2 (\bm y - \bm\mu)^\top \bm A
\bm V_0^{-1}
\bm M
\bm X
\label{eq:dPWRSSdbeta}
\end{equation}
and,
\begin{equation}
\frac{d \mathrm{PWRSS}}{d \bm u} = 
-2 (\bm y - \bm\mu)^\top \bm A
\bm V_0^{-1}
\bm M
\bm Z \bm\Lambda_\theta +
2\bm u^\top
\label{eq:dPWRSSdu}
\end{equation}
Comparing Eqs. \ref{eq:dPWRSSdbeta} and \ref{eq:dPWRSSdu} with Eqs. \ref{eq:dPDEVdbeta} and \ref{eq:dPDEVdu} we see that if the intial guess at $\bm
V$ is correct, then the two sets of equations have the same fixed
points. Note that this result does not imply that PWRSS = PDEV. However, it does suggest an iterative scheme for finding the
fixed effects parameters and the conditional modes. We use the Gauss-Newton
method to derive such an iterative scheme. To do so, we use the techniques
of pseudo-data and weighted residuals to express the problem of
minimizing PWRSS as a non-linear least-squares problem.

\subsection{Pseudo-data and weighted residuals}

In Eq. \ref{eq:PWRSS} the response and its expectation are $\bm\mu$ and $\bm
y$. Using the pseudo-data technique (standard in the mixed model
literature) and the weighted residuals technique (standard in the
weighted least-squares literature) we replace these vectors with,
\begin{equation}
\rho = 
\begin{bmatrix}
\bm W^{1/2}\bm y \\
\bm 0
\end{bmatrix}
\end{equation}
and fitted response,
\begin{equation}
\nu = 
\begin{bmatrix}
\bm W^{1/2}\bm \mu \\
\bm u
\end{bmatrix}
\end{equation}
The pseudo-data, $\bm 0$, and pseudo-mean, $\bm u$, allow us to remove
the penalty term in Eq. ??.  By weighting $\bm y$ and $\bm\mu$ by the
square root of the weights matrix, we can remove the weights matrix
from the expression for PWRSS. In particular, we now have,
\begin{equation}
\mathrm{PWRSS} = (\bm\rho - \bm\nu)^\top (\bm\rho - \bm\nu)
\end{equation}
Minimizing PWRSS is a non-linear least-squares problem, because the
residuals $\bm\rho - \bm\nu$ depend non-linearly on $\bm\beta$ and
$\bm u$.

\subsection{Gauss-Newton}

The Gauss-Newton method is a standard approach for solving non-linear
least-squares problems (Bates and Watts 1988). This method begins by
making a first-order Taylor series approximation of the expected
values, $\bm\nu$. We compute the derivative of $\bm\nu$ with respect
to both $\bm\beta$ and $\bm u$, under the condition that $\bm W = \bm
A \bm V^{-1}$ is fixed at $\bm W_0$,
\begin{displaymath}
\frac{d \bm\nu}{d \bm\beta} = 
\begin{bmatrix}
\bm W_0^{1/2}\bm M \bm X \\
\bm 0
\end{bmatrix}
\end{displaymath}
and 
\begin{displaymath}
\frac{d \bm\nu}{d \bm u} = 
\begin{bmatrix}
\bm W_0^{1/2}\bm M \bm Z \bm\Lambda_\theta \\
\bm I_q
\end{bmatrix}
\end{displaymath}
Importantly, the Jaccobian matrix $\bm M$ depends on $\bm\beta$ and
$\bm u$, but $\bm W_0$ does not because it is taken as
fixed. In practice, this equation will be used in an iterative
algorithm so the specific value of $\bm W_0$ will be calculated
using the current estimate of $\bm\beta = \bm\beta_0$ and $\bm u  =
\bm u_0$. In particular, $\bm W$ depends on $\bm V$, which depends on
$\bm\mu$, which depends on $\bm\beta$ and $\bm u$. In general, we
denote quantities calculated from the current estimates with a $0$
subscript.

Using these derivatives, the first-order Taylor series approximation
to $\bm\nu$ around $(\bm u, \bm\beta) = (\bm u_0, \bm\beta_0)$ is,
\begin{equation}
\bm\nu \approx
\begin{bmatrix}
\bm W_0^{1/2}\bm \mu_0 \\
\bm u_0
\end{bmatrix} + 
\begin{bmatrix}
\bm W_0^{1/2}\bm M_0 \bm Z \bm\Lambda_\theta & \bm W_0^{1/2}\bm M_0 \bm X \\
\bm I_q & \bm 0
\end{bmatrix}
\begin{bmatrix}
\bm u - \bm u_0 \\
\bm\beta - \bm\beta_0 
\end{bmatrix}
\end{equation}
which leads to an approximation of the residuals,
\begin{equation}
\bm\rho - \bm\nu \approx
\begin{bmatrix}
\bm W_0^{1/2}(\bm y - \bm\mu_0) \\
-\bm u_0
\end{bmatrix} - 
\begin{bmatrix}
\bm W_0^{1/2}\bm M_0 \bm Z \bm\Lambda_\theta & \bm W_0^{1/2}\bm M_0 \bm X \\
\bm I_q & \bm 0
\end{bmatrix}
\begin{bmatrix}
\bm u - \bm u_0 \\
\bm\beta - \bm\beta_0 
\end{bmatrix}
\end{equation}
We can simplify this approximation by defining the weighted working
response,
\begin{equation}
\bm r_0 = \bm W_0^{1/2} \bm M_0 (\bm M_0^{-1} (\bm y - \bm\mu_0) +
(\bm\eta_0 - \bm o))
\label{eq:weightedworkingresiduals}
\end{equation}
the weighted working random effects design matrix,
\begin{equation}
\bm U_0 = \bm W_0^{1/2} \bm M_0 \bm Z \bm\Lambda_\theta
\end{equation}
and the weighted working fixed effects design matrix,
\begin{equation}
\bm V_0 = \bm W_0^{1/2} \bm M_0 \bm X
\end{equation}
The resulting simplification is,
\begin{equation}
\bm\rho - \bm\nu \approx
\begin{bmatrix}
\bm r_0 \\
\bm 0
\end{bmatrix} - 
\begin{bmatrix}
\bm U_0 & V_0 \\
\bm I_q & \bm 0
\end{bmatrix}
\begin{bmatrix}
\bm u \\
\bm\beta
\end{bmatrix}
\end{equation}
The normal equations for these residuals are,
\begin{equation}
\begin{bmatrix}
\bm U_0^\top \bm r_0 \\
\bm V_0^\top \bm r_0
\end{bmatrix} = 
\begin{bmatrix}
\bm U_0^\top \bm U_0 + \bm I_q & \bm U_0^\top \bm V_0 \\
\bm V_0^\top \bm U_0 & \bm V_0^\top \bm V_0
\end{bmatrix}
\begin{bmatrix}
\bm u \\
\bm\beta
\end{bmatrix}
\label{eq:normal}
\end{equation}

\subsection{Cholesky decomposition}

To solve the normal equations, we take a Cholesky decomposition of the
cross-product matrix,
\begin{equation}
\begin{bmatrix}
\bm U_0^\top \bm U_0 + \bm I_q & \bm U_0^\top \bm V_0 \\
\bm V_0^\top \bm U_0 & \bm V_0^\top \bm V_0
\end{bmatrix} =
\begin{bmatrix}
\bm L_\theta & \bm 0 \\
\bm R^\top_{ZX} & \bm R^\top_{X} \\
\end{bmatrix}
\begin{bmatrix}
\bm L^\top_\theta & \bm R_{ZX} \\
\bm 0 & \bm R_{X} \\
\end{bmatrix}
\end{equation}
Following the standard theory of Cholesky decompositions we solve
Eq. \ref{eq:normal} in two steps by first solving,
\begin{equation}
\begin{bmatrix}
\bm L_\theta & \bm 0 \\
\bm R^\top_{ZX} & \bm R^\top_{X} \\
\end{bmatrix}
\begin{bmatrix}
\bm c_u \\
\bm c_\beta
\end{bmatrix} = 
\begin{bmatrix}
\bm U_0^\top \bm r_0 \\
\bm V_0^\top \bm r_0 \\
\end{bmatrix}
\end{equation}
for $\bm c_u$ and $\bm c_\beta$, and then solving,
\begin{equation}
\begin{bmatrix}
\bm L^\top_\theta & \bm R_{ZX} \\
\bm 0 & \bm R_{X} \\
\end{bmatrix}
\begin{bmatrix}
\bm u \\
\bm\beta
\end{bmatrix} = 
\begin{bmatrix}
\bm c_u \\
\bm c_\beta
\end{bmatrix}
\end{equation}
for $\bm u$ and $\bm\beta$. In practice we use sparse matrix
representations for $\bm U_0$ and $\bm L_\theta$.

\section{Appendix: PIRLS in abstract terms}

\begin{enumerate}
\setcounter{enumi}{-1}
\item Initialize $\bm\mu_0$ and $\bm\eta_0$
\item Update $\bm W_0$ and $\bm M_0$ using $\bm\mu_0$ and $\bm\eta_0$
\item Update $\bm U_0$ and $\bm V_0$ using $\bm W_0$ and $\bm M_0$
\item Update $\bm L_\theta$, $\bm R_{ZX}$, and $\bm R_{X}$ using $\bm U_0$ and $\bm V_0$
\item Update $\bm r_0$, using $\bm W_0$, $\bm M_0$, $\bm\mu_0$, and $\bm\eta_0$
\item Solve the normal equations for $\bm u$ and $\bm\beta$ using
  $\bm L_\theta$, $\bm R_{ZX}$, $\bm R_{X}$, and $\bm r_0$
\item Update $\bm u_0$ and $\bm\beta_0$ to these solutions
\item Update $\bm\mu_0$ and $\bm\eta_0$ using $\bm u_0$ and
  $\bm\beta_0$
\item Compute PDEV
  using $\bm\mu_0$, $\bm y$, and $\bm a$
\begin{itemize}
\item If the deviance does not decrease, try step halving
\item If step-halving fails, return an error
\end{itemize}
\item Repeat 1-8 until convergence
\begin{itemize}
\item If convergence is reached, return estimates of $\bm u$ and $\bm\beta$
\item If the convergence limit is reached, return an error
\end{itemize}
\end{enumerate}

\section{Appendix: PIRLS (nAGQ > 0 version)}

A more accurate algorithm is obtained if use PIRLS to estimate $u$
only, and numerically optimize $\beta$ along with $\theta$. PIRLS
with $u$ only is obtained by moving the influence of the fixed
effect coefficients, $\beta$, from the weighted design matrix to the weighted
working response. In particular, the weighted working residuals
(Eq. \ref{eq:weightedworkingresiduals}) become,
\begin{equation}
\bm r_0 = \bm W_0^{1/2} \bm M_0 (\bm M_0^{-1} (\bm y - \bm\mu_0) +
(\bm\eta_0 - \bm o - \bm X \bm\beta))
\end{equation}
and the normal equations (Eq. \ref{eq:normal}) become,
\begin{equation}
\bm U_0^\top \bm r_0 = 
(\bm U_0^\top \bm U_0 + \bm I_q ) \bm u
\end{equation}
The Cholesky decomposition now simplifies to,
\begin{equation}
\bm U_0^\top \bm U_0 + \bm I_q =
\bm L_\theta \bm L^\top_\theta 
\end{equation}
and the two-step solution to the normal equations simplies to first solving,
\begin{equation}
\bm L_\theta
\bm c_u = 
\bm U_0^\top \bm r_0
\end{equation}
for $\bm c_u$, and then solving,
\begin{equation}
\bm L^\top_\theta
\bm u= 
\bm c_u
\end{equation}
for $\bm u$. Again, in practice we use sparse matrix
representations for $\bm U_0$ and $\bm L_\theta$.

\section{Appendix: IRLS}

\begin{equation}
\bm\nu \approx
\bm W_0^{1/2}\bm \mu_0 + 
 \bm W_0^{1/2}\bm M_0 \bm X (\bm\beta - \bm\beta_0 )
\end{equation}
which leads to an approximation of the residuals,
\begin{equation}
\bm\rho - \bm\nu \approx
\bm W_0^{1/2}(\bm y - \bm\mu_0) - 
\bm W_0^{1/2}\bm M_0 \bm X 
(\bm\beta - \bm\beta_0)
\end{equation}
\begin{equation}
\bm\rho - \bm\nu \approx
\bm W_0^{1/2}((\bm y - \bm\mu_0) + 
\bm M_0 (\bm\eta_0 - \bm o)) - 
\bm W_0^{1/2}\bm M_0 \bm X \bm\beta
\end{equation}

\bibliography{lmer}
\end{document}

%%% Local Variables:
%%% mode: latex
%%% TeX-master: t
%%% End:
