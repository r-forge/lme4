% NOTE -- ONLY EDIT THE .Rnw FILE!!!  The .tex file is
% likely to be overwritten.

\usepackage{SweaveSlides}
\title[lme4]{Mixed models in R using the lme4 package\\Part 3: Linear mixed models with simple, scalar random effects}
\subject{LMM}
\date[July 1, 2009]{University of Lausanne\\July 1, 2009}
\AtBeginSection[]
{
  \begin{frame}<beamer>
    \frametitle{Outline}
    \tableofcontents[currentsection]
  \end{frame}
}
\begin{document}
\frame{\titlepage}
\begin{frame}
  \frametitle{Outline}
  \tableofcontents[pausesections,hideallsubsections]
\end{frame}

\newcommand{\bc}[1]{\ensuremath{\bm{\mathcal{#1}}}}


\setkeys{Gin}{width=\textwidth}

\section[Packages]{R packages and data in packages}

\begin{frame}[fragile]
  \frametitle{R packages}
  \begin{itemize}
  \item Packages incorporate functions, data and documentation.
  \item You can produce packages for private or in-house use or you
    can contribute your package to the Comprehensive R Archive Network
    (CRAN), \url{http://cran.R-project.org}
  \item We will be using the \Emph{lme4} package from CRAN.
    Install it from the \Emph{Packages} menu item or with
\begin{Schunk}
\begin{Sinput}
> install.packages("lme4")
\end{Sinput}
\end{Schunk}
\item You only need to install a package once.  If a new version
  becomes available you can update (see the menu item).
\item To use a package in an R session you attach it using
\begin{Schunk}
\begin{Sinput}
> require(lme4)
\end{Sinput}
\end{Schunk}
  or
\begin{Schunk}
\begin{Sinput}
> library(lme4)
\end{Sinput}
\end{Schunk}
  (This usage causes widespread confusion of the terms ``package'' and ``library''.)
  \end{itemize}
\end{frame}

\begin{frame}[fragile]
\frametitle{Accessing documentation}
\begin{itemize}
  \item To be added to CRAN, a package must pass a series of quality
    control checks.  In particular, all functions and data sets must
    be documented.  Examples and tests can also be included.
  \item The \code{data} function provides names and brief descriptions
    of the data sets in a package.
    \begin{Schunk}
      \begin{Sinput}
> data(package = "lme4")        
      \end{Sinput}
      \begin{Soutput}
Data sets in package 'lme4':

Dyestuff                Yield of dyestuff by batch
Dyestuff2               Yield of dyestuff by batch
Pastes                  Paste strength by batch and cask
Penicillin              Variation in penicillin testing
cake                    Breakage angle of chocolate cakes
cbpp                    Contagious bovine pleuropneumonia
sleepstudy              Reaction times in a sleep deprivation study
      \end{Soutput}
    \end{Schunk}
  \item Use \code{?} followed by the name of a function or data set to
    view its documentation.  If the documentation contains an example
    section, you can execute it with the \code{example} function.
\end{itemize}
\end{frame}

\begin{frame}
  \frametitle{Effects - fixed and random}
  \begin{itemize}
  \item Mixed-effects models, like many statistical models, describe
    the relationship between a \Emph{response} variable and one or
    more \Emph{covariates} recorded with it.
  \item The models we will discuss are based on a \Emph{linear
      predictor} expression incorporating \Emph{coefficients} that are
    estimated from the observed data.
  \item Coefficients associated with the levels of a categorical
    covariate are sometimes called the \Emph{effects} of the levels.
  \item When the levels of a covariate are fixed and reproducible
    (e.g. a covariate \code{sex} that has levels \code{male} and
    \code{female}) we incorporate them as fixed-effects parameters.
  \item When the levels of a covariate correspond to the particular
    observational or experimental units in the experiment we
    incorporate them as \code{random effects}.
  \end{itemize}
\end{frame}

\section[Dyestuff]{The Dyestuff data and model}

\begin{frame}[fragile]
  \frametitle{The Dyestuff data set}
\begin{itemize}
\item The \code{Dyestuff}, \code{Penicillin} and \code{Pastes} data
  sets all come from the classic book \Emph{Statistical Methods in
  Research and Production}, edited by O.L. Davies  and first published
  in 1947.
\item The \code{Dyestuff} data are a balanced one-way classification
  of the \code{Yield} of dyestuff from samples produced from six
  \code{Batch}es of an intermediate product. See \code{?Dyestuff}.
\end{itemize}
\begin{Schunk}
\begin{Sinput}
> str(Dyestuff)
\end{Sinput}
\begin{Soutput}
'data.frame':	30 obs. of  2 variables:
 $ Batch: Factor w/ 6 levels "A","B","C","D",..: 1 1 1 1 1 2 2 2 2 2 ...
 $ Yield: num  1545 1440 1440 1520 1580 ...
\end{Soutput}
\begin{Sinput}
> summary(Dyestuff)
\end{Sinput}
\begin{Soutput}
 Batch     Yield     
 A:5   Min.   :1440  
 B:5   1st Qu.:1469  
 C:5   Median :1530  
 D:5   Mean   :1528  
 E:5   3rd Qu.:1575  
 F:5   Max.   :1635  
\end{Soutput}
\end{Schunk}
\end{frame}

\begin{frame}
  \frametitle{The effect of the batches}
  \begin{itemize}
  \item To emphasize that \code{Batch} is categorical, we use letters
    instead of numbers to designate the levels.
  \item Because there is no inherent ordering of the levels of
    \code{Batch}, we will reorder the levels if, say, doing so can
    make a plot more informative.
  \item The particular batches observed are just a selection of the
    possible batches and are entirely used up during the course of
    the experiment.  
  \item It is not particularly important to estimate and compare
    yields from these batches.  Instead we wish to estimate the
    variability in yields due to batch-to-batch variability.
  \item The \code{Batch} factor will be used in \Emph{random-effects}
    terms in models that we fit.
  \end{itemize}
\end{frame}

\begin{frame}[fragile]
  \frametitle{Dyestuff data plot}
  \begin{center}
\includegraphics{figs/data-Dyestuffplot}
  \end{center}
  \begin{itemize}
  \item The line joins the mean yields of the six batches, which have
    been reordered by increasing mean yield.
  \item The vertical positions are jittered slightly to reduce
    overplotting.  The lowest yield for batch A was observed on two
    distinct preparations from that batch.
  \end{itemize}
\end{frame}

\begin{frame}[fragile]
  \frametitle{A mixed-effects model for the dyestuff yield}
\begin{Schunk}
\begin{Sinput}
> fm1 <- lmer(Yield ~ 1 + (1 | Batch), Dyestuff)
> print(fm1)
\end{Sinput}
\begin{Soutput}
Linear mixed model fit by REML 
Formula: Yield ~ 1 + (1 | Batch) 
   Data: Dyestuff 
   AIC   BIC logLik deviance REMLdev
 325.7 329.9 -159.8    327.4   319.7
Random effects:
 Groups   Name        Variance Std.Dev.
 Batch    (Intercept) 1764.0   42.001  
 Residual             2451.3   49.510  
Number of obs: 30, groups: Batch, 6
Fixed effects:
            Estimate Std. Error t value
(Intercept)  1527.50      19.38   78.81
\end{Soutput}
\end{Schunk}
\begin{itemize}
\item Fitted model \code{fm1} has one fixed-effect parameter, the mean
  yield, and one random-effects term, generating a simple, scalar
  random effect for each level of \code{Batch}.
\end{itemize}
\end{frame}


\begin{frame}[fragile]
  \frametitle{Extracting information from the fitted model}
  \begin{itemize}
  \item \code{fm1} is an object of class \code{"mer"} (mixed-effects
    representation). 
  \item There are many \Emph{extractor} functions that can be applied
    to such objects.
  \end{itemize}
\begin{Schunk}
\begin{Sinput}
> fixef(fm1)
\end{Sinput}
\begin{Soutput}
(Intercept) 
     1527.5 
\end{Soutput}
\begin{Sinput}
> ranef(fm1, drop = TRUE)
\end{Sinput}
\begin{Soutput}
$Batch
        A         B         C         D         E         F 
-17.60800   0.39129  28.56409 -23.08605  56.73689 -44.99823 
\end{Soutput}
\begin{Sinput}
> fitted(fm1)
\end{Sinput}
\begin{Soutput}
 [1] 1509.9 1509.9 1509.9 1509.9 1509.9 1527.9 1527.9 1527.9 1527.9
[10] 1527.9 1556.1 1556.1 1556.1 1556.1 1556.1 1504.4 1504.4 1504.4
[19] 1504.4 1504.4 1584.2 1584.2 1584.2 1584.2 1584.2 1482.5 1482.5
[28] 1482.5 1482.5 1482.5
\end{Soutput}
\end{Schunk}
\end{frame}


\section[Mixed models]{Definition of mixed-effects models}

\begin{frame}[fragile]
  \frametitle{Definition of mixed-effects models}
  \begin{itemize}
  \item Models with random effects are often written like
    \begin{displaymath}
      y_{ij}=\mu+b_i+\epsilon_{ij},\;
      b_i\sim\mathcal{N}(0,\sigma_b^2),
      \epsilon_{ij}\sim\mathcal{N}(0,\sigma^2),i=1,\dots,I,j=1,\dots,J_i
    \end{displaymath}
  \item This scalar notation quickly becomes unwieldy, degenerating
    into ``subscript fests''. We will use a vector/matrix notation.
  \item A mixed-effects model incorporates two vector-valued random
    variables: the response vector, $\bc{Y}$, and the random effects
    vector, $\bc{B}$. We observe the value, $\bm y$, of $\bc{Y}$.  We
    do not observe the value of $\bc{B}$.
  \item In the models we will consider, the random effects are modeled
    as a multivariate Gaussian (or ``normal'') random variable,
    $\bc{B}\sim\mathcal{N}(\bm 0,\bm\Sigma(\bm\theta))$, where
    $\bm\theta$ is a vector of \Emph{variance-component parameters}.
  \end{itemize}
\end{frame}

\begin{frame}
  \frametitle{Linear mixed models}
  \begin{itemize}
  \item The conditional distribution, $(\bc Y|\bc B=\bm b)$, depends on
    $\bm b$ only through its mean, $\bm\mu_{\bc Y|\bc B=\bm b}$.
  \item The conditional mean, $\bm\mu_{\bc Y|\bc B=\bm b}$, depends on
    $\bm b$ and on the fixed-effects parameter vector, $\bm\beta$, through a
    \Emph{linear predictor} expression, $\bm Z\bm b+\bm X\bm\beta$.
    The \Emph{model matrices} $\bm Z$ and $\bm X$ are determined from the
    form of the model and the values of the covariates.
  \item In a \Emph{linear mixed model} the conditional distribution is
    a ``spherical'' multivariate Gaussian
    \begin{displaymath}
      (\bc{Y}|\bc{B}=\bm b)\sim\mathcal{N}(\bm Z\bm b+\bm X\bm\beta,
      \sigma^2\bm I_n)
    \end{displaymath}
  \item The scalar $\sigma$ is the \Emph{common scale parameter}; the
    dimension of $\bm y$ is $n$, $\bm b$ is $q$ and $\bm\beta$ is $p$
    so $\bm Z$ is $n\times q$ and $\bm X$ is $n\times p$.
  \end{itemize}
\end{frame}

\begin{frame}
  \frametitle{Simple, scalar random effects terms}
  \begin{itemize}
  \item A term like \code{(1|Batch)} in an \code{lmer} formula is
    called a simple, scalar random-effects term.
  \item The expression on the right of the \code{"|"} operator
    (usually just the name of a variable) is evaluated as a factor,
    called the \Emph{grouping factor} for the term.
  \item Suppose we have $k$ such terms with $n_i,i=1,\dots,k$ levels
    in the $i$th term's grouping factor. A scalar random-effects term
    generates one random effect for each level of the grouping factor.
    If all the random effects terms are scalar terms then
    $q=\sum_{i=1}^kn_i$.
  \item The model matrix $\bm Z$ is the horizontal concatenation of $k$
    matrices.  For a simple, scalar term, the $i$th vertical slice,
    which has $n_i$ columns, is the indicator columns for the $n_i$
    levels of the $i$th grouping factor.
  \end{itemize}
\end{frame}

\begin{frame}
  \frametitle{Structure of the unconditional variance-covariance}
  \begin{itemize}
  \item Just as the matrix $\bm Z$ is the horizontal concatenation of
    matrices generated by individual random-effects terms, the
    (unconditional) variance-covariance matrix, $\bm\Sigma$, is
    block-diagonal in $k$ blocks.  In other words, the unconditional
    distributions of random effects from different terms in the model
    are independent.  (However, the conditional distributions, given
    the observed data, $(\bc B|\bc Y=\bm y)$, are not independent.)
  \item If the $i$th term is a simple, scalar term then the $i$th
    diagonal block is a multiple of the identity, $\sigma_i^2\bm I_{n_i}$.
  \item This means that unconditional distributions of random effects
    corresponding to different levels of the grouping factor are
    independent.
  \end{itemize}
\end{frame}

\begin{frame}[fragile]
  \frametitle{Model matrices for model fm1}
  \begin{itemize}
  \item The formula for model \code{fm1} has a single fixed-effects
    term, \code{1}, and one simple, scalar random-effects term,
    \code{(1|Batch)}.
  \item The model matrix, $\bm Z$, whose transpose is stored in a slot
    called \code{Zt}, is the matrix of indicators for the six levels
    of \code{Batch}.
  \item The model matrix, $\bm X$, is $30\times 1$. All its elements
    are unity.
  \end{itemize}
\begin{Schunk}
\begin{Sinput}
> str(model.matrix(fm1))
\end{Sinput}
\begin{Soutput}
 num [1:30, 1] 1 1 1 1 1 1 1 1 1 1 ...
 - attr(*, "assign")= int 0
\end{Soutput}
\begin{Sinput}
> fm1@Zt
\end{Sinput}
\begin{Soutput}
6 x 30 sparse Matrix of class "dgCMatrix"
[1,] 1 1 1 1 1 . . . . . . . . . . . . . . . . . . . . . . . . .
[2,] . . . . . 1 1 1 1 1 . . . . . . . . . . . . . . . . . . . .
[3,] . . . . . . . . . . 1 1 1 1 1 . . . . . . . . . . . . . . .
[4,] . . . . . . . . . . . . . . . 1 1 1 1 1 . . . . . . . . . .
[5,] . . . . . . . . . . . . . . . . . . . . 1 1 1 1 1 . . . . .
[6,] . . . . . . . . . . . . . . . . . . . . . . . . . 1 1 1 1 1
\end{Soutput}
\end{Schunk}
\end{frame}

\begin{frame}
  \frametitle{Conditional means of the random effects}
  \begin{itemize}
  \item Technically we do not provide ``estimates'' of the random
    effects because they are not parameters.  
  \item One answer to the question, ``so what are those numbers
    provided by \code{ranef} anyway?'' is that they are BLUPs (Best
    Linear Unbiased Predictors) of the random effects.  The acronym is
    attractive but not very informative (what is a ``linear unbiased
    predictor'' and in what sense are these the ``best''?).  Also, the
    concept does not generalize.
  \item A better answer is that those values are the conditional
    means, $\bm\mu_{\bc B|\bc Y =\bm y}$, evaluated at the estimated
    parameter values.  Regrettably, we can only evaluate the
    conditional means for linear mixed models.
  \item However, these values are also the conditional modes and that
    concept does generalize to other types of mixed models.
  \end{itemize}
\end{frame}

\begin{frame}[fragile]
  \frametitle{Caterpillar plot for fm1}
  \begin{itemize}
  \item For linear mixed models the conditional distribution of the
    random effects, given the data, written $(\bc B|\bc Y=\bm y)$, is
    again a multivariate Gaussian distribution.
  \item We can evaluate the means and standard deviations of the
    individual conditional distributions, $(\bc B_j|\bc Y=\bm y), j =
    1,\dots,q$.  We show these in the form of a 95\% prediction
    interval, with the levels of the grouping factor arranged in
    increasing order of the conditional mean.
  \item These are sometimes called ``caterpillar plots''.
  \end{itemize}
  \begin{center}
\includegraphics{figs/data-fm1ranef}
  \end{center}
\end{frame}

\begin{frame}
  \frametitle{REML estimates versus ML estimates}
  \begin{itemize}
  \item The default parameter estimation criterion for linear mixed
    models is restricted (or ``residual'') maximum likelihood (REML).
  \item Maximum likelihood (ML) estimates (sometimes called ``full
    maximum likelihood'') can be requested by specifying \code{REML =
      FALSE} in the call to \code{lmer}.
  \item Generally REML estimates of variance components are
    preferred.  ML estimates are known to be biased.  Although REML
    estimates are not guaranteed to be unbiased, they are usually less
    biased than ML estimates.
  \item Roughly, the difference between REML and ML estimates of
    variance components is comparable to estimating $\sigma^2$ in a
    fixed-effects regression by $\mathit{SSR}/(n-p)$ versus
    $\mathit{SSR}/n$, where $\mathit{SSR}$ is the residual sum of
    squares.
  \item For a balanced, one-way classification like the
    \code{Dyestuff} data, the REML and ML estimates of the fixed-effects
    are identical.
  \end{itemize}
\end{frame}

\begin{frame}[fragile]
  \frametitle{Re-fitting the model for ML estimates}
\begin{Schunk}
\begin{Sinput}
> (fm1M <- update(fm1, REML = FALSE))
\end{Sinput}
\begin{Soutput}
Linear mixed model fit by maximum likelihood 
Formula: Yield ~ 1 + (1 | Batch) 
   Data: Dyestuff 
   AIC   BIC logLik deviance REMLdev
 333.3 337.5 -163.7    327.3   319.7
Random effects:
 Groups   Name        Variance Std.Dev.
 Batch    (Intercept) 1388.3   37.26   
 Residual             2451.3   49.51   
Number of obs: 30, groups: Batch, 6
Fixed effects:
            Estimate Std. Error t value
(Intercept)  1527.50      17.69   86.33
\end{Soutput}
\end{Schunk}

(The extra parentheses around the assignment cause the value to
be printed.  Generally the results of assignments are not printed.)

\end{frame}

\begin{frame}
  \frametitle{Verbose fitting}
  \begin{itemize}
  \item When fitting a large model or if the estimates of the variance
    components seem peculiar, it is a good idea to monitor the
    progress of the iterations in optimizing the deviance or the REML
    criterion.
  \item The optional argument \code{verbose = TRUE} causes \code{lmer}
    to print iteration information during the optimzation of the
    parameter estimates.
  \item The quantity being minimized is the \Emph{profiled deviance}
    or the \Emph{profiled REML criterion} of the model.  The deviance
    is negative twice the log-likelihood.  It is profiled in the sense
    that it is a function of $\bm\theta$ only --- $\bm\beta$ and
    $\sigma$ are at their conditional estimates.
  \end{itemize}
\end{frame}

\begin{frame}[fragile]
  \frametitle{Obtain the verbose output for fitting fm1}
\begin{Schunk}
\begin{Sinput}
> invisible(update(fm1, verbose = TRUE))
\end{Sinput}
\begin{Soutput}
  0:     319.76562: 0.730297
  1:     319.73549: 0.962389
  2:     319.65735: 0.869461
  3:     319.65441: 0.844025
  4:     319.65428: 0.848469
  5:     319.65428: 0.848327
  6:     319.65428: 0.848324
\end{Soutput}
\end{Schunk}
\begin{itemize}
\item The first number on each line is the iteration count --- iteration
  0 is at the starting value for $\bm\theta$.
\item The second number is the profiled deviance or profiled REML
  criterion --- the quantity being minimized.
\item The third and subsequent numbers are the parameter vector $\bm\theta$.
\end{itemize}
\end{frame}

\begin{frame}[fragile]
  \frametitle{Estimates of variance components can be zero}
  \begin{itemize}
  \item We have been careful to state the variance of the random
    effects is $\ge0$.
  \item For some data sets the maximum likelihood or REML estimate,
    $\widehat{\sigma_b^2}$ is zero.
  \item Box and Tiao (1973) provide simulated data with a structure
    like the \code{Dyestuff} data illustrating this.
  \end{itemize}
\begin{Schunk}
\begin{Sinput}
> str(Dyestuff2)
\end{Sinput}
\begin{Soutput}
'data.frame':	30 obs. of  2 variables:
 $ Batch: Factor w/ 6 levels "A","B","C","D",..: 1 1 1 1 1 2 2 2 2 2 ...
 $ Yield: num  7.3 3.85 2.43 9.57 7.99 ...
\end{Soutput}
\end{Schunk}
\end{frame}
\begin{frame}[fragile]
  \frametitle{Plot of the Dyestuff2 data}
  \begin{center}
\includegraphics{figs/data-Dyestuff2plot}
  \end{center}
  \begin{itemize}
  \item For these data the batch-to-batch variability is not large
    compared to the within-batch variability.
  \end{itemize}
\end{frame}
\begin{frame}[fragile]
  \frametitle{Fitting the model to Dyestuff2}
\begin{Schunk}
\begin{Sinput}
> (fm1A <- lmer(Yield ~ 1 + (1 | Batch), Dyestuff2, verbose = TRUE))
\end{Sinput}
\begin{Soutput}
  0:     166.04147: 0.730297
  1:     161.82828:  0.00000
  2:     161.82828:  0.00000
Linear mixed model fit by REML 
Formula: Yield ~ 1 + (1 | Batch) 
   Data: Dyestuff2 
   AIC   BIC logLik deviance REMLdev
 167.8 172.0 -80.91    162.9   161.8
Random effects:
 Groups   Name        Variance Std.Dev.
 Batch    (Intercept)  0.000   0.0000  
 Residual             13.806   3.7157  
Number of obs: 30, groups: Batch, 6
Fixed effects:
            Estimate Std. Error t value
(Intercept)   5.6656     0.6784   8.352
\end{Soutput}
\end{Schunk}
\end{frame}
\begin{frame}[fragile]
  \frametitle{A trivial mixed-effects model is a fixed-effects model}
  \begin{itemize}
  \item The mixed model \code{fm1A} with an estimated variance
    $\widehat{\sigma_b^2}=0$ is equivalent to a model with only
    fixed-effects terms.
  \end{itemize}
\begin{Schunk}
\begin{Sinput}
> summary(lm1 <- lm(Yield ~ 1, Dyestuff2))
\end{Sinput}
\begin{Soutput}
Call:
lm(formula = Yield ~ 1, data = Dyestuff2)
Residuals:
    Min      1Q  Median      3Q     Max 
-6.5576 -2.9006 -0.3006  2.4854  7.7684 

Coefficients:
            Estimate Std. Error t value Pr(>|t|)
(Intercept)   5.6656     0.6784   8.352 3.32e-09

Residual standard error: 3.716 on 29 degrees of freedom
\end{Soutput}
\begin{Sinput}
> logLik(lm1)
\end{Sinput}
\begin{Soutput}
'log Lik.' -81.43652 (df=2)
\end{Soutput}
\end{Schunk}
\end{frame}
\begin{frame}[fragile]
  \frametitle{Recap of the Dyestuff model}
  \begin{itemize}
  \item The model is fit as
\begin{Schunk}
\begin{Soutput}
lmer(formula = Yield ~ 1 + (1 | Batch), data = Dyestuff)
\end{Soutput}
\end{Schunk}
\item There is one random-effects term, \code{(1|Batch)}, in the model
  formula.  It is a simple, scalar term for the grouping factor
  \code{Batch} with $n_1=6$ levels.  Thus $q=6$.
\item The model matrix $\bm Z$ is the $30\times 6$ matrix of
  indicators of the levels of \code{Batch}.
\item The relative variance-covariance matrix, $\bm\Sigma$, is a
  nonnegative multiple of the $6\times 6$ identity matrix $\bm I_6$.
\item The fixed-effects parameter vector, $\bm\beta$, is of length
  $p=1$.  All the elements of the $30\times 1$ model matrix $\bm X$
  are unity.
  \end{itemize}
\end{frame}

\section[Penicillin]{Crossed random-effects grouping: Penicillin}
\begin{frame}[fragile]
  \frametitle{The Penicillin data (see also the ?Penicillin description)}
\begin{Schunk}
\begin{Sinput}
> str(Penicillin)
\end{Sinput}
\begin{Soutput}
'data.frame':	144 obs. of  3 variables:
 $ diameter: num  27 23 26 23 23 21 27 23 26 23 ...
 $ plate   : Factor w/ 24 levels "a","b","c","d",..: 1 1 1 1 1 1 2 2 2 2 ...
 $ sample  : Factor w/ 6 levels "A","B","C","D",..: 1 2 3 4 5 6 1 2 3 4 ...
\end{Soutput}
\begin{Sinput}
> xtabs(~sample + plate, Penicillin)
\end{Sinput}
\begin{Soutput}
      plate
sample a b c d e f g h i j k l m n o p q r s t u v w x
     A 1 1 1 1 1 1 1 1 1 1 1 1 1 1 1 1 1 1 1 1 1 1 1 1
     B 1 1 1 1 1 1 1 1 1 1 1 1 1 1 1 1 1 1 1 1 1 1 1 1
     C 1 1 1 1 1 1 1 1 1 1 1 1 1 1 1 1 1 1 1 1 1 1 1 1
     D 1 1 1 1 1 1 1 1 1 1 1 1 1 1 1 1 1 1 1 1 1 1 1 1
     E 1 1 1 1 1 1 1 1 1 1 1 1 1 1 1 1 1 1 1 1 1 1 1 1
     F 1 1 1 1 1 1 1 1 1 1 1 1 1 1 1 1 1 1 1 1 1 1 1 1
\end{Soutput}
\end{Schunk}
\begin{itemize}
\item These are measurements of the potency (measured by the diameter
  of a clear area on a Petri dish) of penicillin samples in a
  balanced, unreplicated two-way crossed classification with the test
  medium, \code{plate}.
\end{itemize}
\end{frame}

\begin{frame}[fragile]
  \frametitle{Penicillin data plot}
  \begin{center}
\includegraphics{figs/data-PenicillinPlot}
  \end{center}
\end{frame}

\begin{frame}[fragile]
  \frametitle{Model with crossed simple random effects for Penicillin}
\begin{Schunk}
\begin{Sinput}
> (fm2 <- lmer(diameter ~ 1 + (1 | plate) + (1 | sample), 
+     Penicillin))
\end{Sinput}
\begin{Soutput}
Linear mixed model fit by REML 
Formula: diameter ~ 1 + (1 | plate) + (1 | sample) 
   Data: Penicillin 
   AIC   BIC logLik deviance REMLdev
 338.9 350.7 -165.4    332.3   330.9
Random effects:
 Groups   Name        Variance Std.Dev.
 plate    (Intercept) 0.71691  0.84670 
 sample   (Intercept) 3.73092  1.93156 
 Residual             0.30242  0.54992 
Number of obs: 144, groups: plate, 24; sample, 6
Fixed effects:
            Estimate Std. Error t value
(Intercept)  22.9722     0.8085   28.41
\end{Soutput}
\end{Schunk}
\end{frame}


\begin{frame}[fragile]
  \frametitle{Fixed and random effects for fm2}
  \begin{itemize}
  \item The model for the $n=144$ observations has $p=1$ fixed-effects
    parameter and $q=30$ random effects from $k=2$ random effects
    terms in the formula.
  \end{itemize}
\begin{Schunk}
\begin{Sinput}
> fixef(fm2)
\end{Sinput}
\begin{Soutput}
(Intercept) 
     22.972 
\end{Soutput}
\begin{Sinput}
> ranef(fm2, drop = TRUE)
\end{Sinput}
\begin{Soutput}
$plate
        a         b         c         d         e         f 
 0.804547  0.804547  0.181672  0.337391  0.025953 -0.441203 
        g         h         i         j         k         l 
-1.375516  0.804547 -0.752641 -0.752641  0.960266  0.493109 
        m         n         o         p         q         r 
 1.427422  0.493109  0.960266  0.025953 -0.285484 -0.285484 
        s         t         u         v         w         x 
-1.375516  0.960266 -0.908360 -0.285484 -0.596922 -1.219797 
$sample
        A         B         C         D         E         F 
 2.187245 -1.010563  1.938065 -0.096903 -0.013843 -3.004001 
\end{Soutput}
\end{Schunk}
\end{frame}


\begin{frame}[fragile]
  \frametitle{Prediction intervals for random effects}
\includegraphics{figs/data-fm2ranef}
\end{frame}

\begin{frame}[fragile]
  \frametitle{Model matrix Z for fm2}
  \begin{itemize}
  \item Because the model matrix $\bm Z$ is generated from $k=2$
    simple, scalar random effects terms, it consists of two sets of
    indicator columns.
  \item The structure of $\bm Z\trans$ is shown below.  (Generally we
    will show the transpose of these model matrices - they fit better
    on slides.)
  \end{itemize}
  \begin{center}
\includegraphics{figs/data-fm2Z}
  \end{center}
\end{frame}

\begin{frame}
  \frametitle{Models with crossed random effects}
  \begin{itemize}
  \item Many people believe that mixed-effects models are equivalent
    to hierarchical linear models (HLMs) or ``multilevel models''.
    This is not true.  The \code{plate} and \code{sample} factors in
    \code{fm2} are crossed.  They do not represent levels in a hierarchy.
  \item There is no difficulty in defining and fitting models with
    crossed random effects (meaning random-effects terms whose
    grouping factors are crossed).  However, fitting models with
    crossed random effects can be somewhat slower.
  \item The crucial calculation in each \code{lmer} iteration is
    evaluation of a $q\times q$ sparse, lower triangular, Cholesky
    factor, $\bm L(\bm\theta)$, derived from $\bm Z$ and
    $\bm\Sigma(\bm\theta)$.  Crossing of grouping factors increases
    the number of nonzeros in $\bm L(\bm\theta)$ and causes some
    ``fill-in'' of $\bm L$ relative to $\bm Z\trans\bm Z$.
  \end{itemize}
\end{frame}

\begin{frame}[fragile]
  \frametitle{All HLMs are mixed models but not vice-versa}
  \begin{itemize}
  \item Even though Raudenbush and Bryk (2002) do discuss models for
    crossed factors in their HLM book, such models are not
    hierarchical.
  \item Experimental situations with crossed random factors, such as
    ``subject'' and ``stimulus'', are common.  We can, and should, model
    such data according to its structure.
  \item In longitudinal studies of subjects in social contexts (e.g.
    students in classrooms or in schools) we almost always have partial
    crossing of the subject and the context factors, meaning that, over
    the course of the study, a particular student may be observed in
    more than one class but not all students are
    observed in all classes.  The student and class factors are
    neither fully crossed nor strictly nested.  
  \item For longitudinal data, ``nested'' is only important if it means
    ``nested across time''.  ``Nested at a particular time'' doesn't
    count.
  \item \code{lme4} handles fully or partially crossed factors gracefully.
  \end{itemize}
\end{frame}

\begin{frame}[fragile]
  \frametitle{Recap of the Penicillin model}
  \begin{itemize}
  \item The model formula is
\begin{Schunk}
\begin{Soutput}
diameter ~ 1 + (1 | plate) + (1 | sample)
\end{Soutput}
\end{Schunk}
\item There are two random-effects terms, \code{(1|plate)} and
  \code{(1|sample)}.  Both are simple, scalar random effects terms,
  with $n_1=24$ and $n_2=6$ levels, respectively.  Thus
  $q=q_1n_1+q_2n_2=30$.
\item The model matrix $\bm Z$ is the $144\times 30$ matrix created
  from two sets of indicator columns.
\item The relative variance-covariance matrix, $\bm\Sigma$, is block
  diagonal in two blocks that are nonnegative multiples of identity
  matrices.
\item The fixed-effects parameter vector, $\bm\beta$, is of length
  $p=1$.  All the elements of the $144\times 1$ model matrix $\bm X$
  are unity.
\end{itemize}
\end{frame}

\section[Pastes]{Nested random-effects grouping: Pastes}

\begin{frame}[fragile]
  \frametitle{The Pastes data (see also the ?Pastes description)}
\begin{Schunk}
\begin{Sinput}
> str(Pastes)
\end{Sinput}
\begin{Soutput}
'data.frame':	60 obs. of  4 variables:
 $ strength: num  62.8 62.6 60.1 62.3 62.7 63.1 60 61.4 57.5 56.9 ...
 $ batch   : Factor w/ 10 levels "A","B","C","D",..: 1 1 1 1 1 1 2 2 2 2 ...
 $ cask    : Factor w/ 3 levels "a","b","c": 1 1 2 2 3 3 1 1 2 2 ...
 $ sample  : Factor w/ 30 levels "A:a","A:b","A:c",..: 1 1 2 2 3 3 4 4 5 5 ...
\end{Soutput}
\begin{Sinput}
> xtabs(~batch + sample, Pastes, sparse = TRUE)
\end{Sinput}
\begin{Soutput}
10 x 30 sparse Matrix of class "dgCMatrix"
A 2 2 2 . . . . . . . . . . . . . . . . . . . . . . . . . . .
B . . . 2 2 2 . . . . . . . . . . . . . . . . . . . . . . . .
C . . . . . . 2 2 2 . . . . . . . . . . . . . . . . . . . . .
D . . . . . . . . . 2 2 2 . . . . . . . . . . . . . . . . . .
E . . . . . . . . . . . . 2 2 2 . . . . . . . . . . . . . . .
F . . . . . . . . . . . . . . . 2 2 2 . . . . . . . . . . . .
G . . . . . . . . . . . . . . . . . . 2 2 2 . . . . . . . . .
H . . . . . . . . . . . . . . . . . . . . . 2 2 2 . . . . . .
I . . . . . . . . . . . . . . . . . . . . . . . . 2 2 2 . . .
J . . . . . . . . . . . . . . . . . . . . . . . . . . . 2 2 2
\end{Soutput}
\end{Schunk}
\end{frame}

\begin{frame}[fragile]
  \frametitle{Structure of the Pastes data}
  \begin{itemize}
  \item The \code{sample} factor is nested within the \code{batch}
    factor. Each sample is from one of three casks selected from a
    particular batch.
  \item Note that there are 30, not 3, distinct samples.
  \item We can label the casks as `a', `b' and `c' but then the
    \code{cask} factor by itself is meaningless (because cask `a' in
    batch `A' is unrelated to cask `a'in batches `B', `C', $\dots$).
    The \code{cask} factor is only meaningful within a \code{batch}.
  \item Only the \code{batch} and \code{cask} factors, which are
    apparently crossed, were present in the original data set.
    \code{cask} may be described as being nested within \code{batch}
    but that is not reflected in the data.  It is \Emph{implicitly
      nested}, not explicitly nested.
  \item You can save yourself a lot of grief by immediately creating
    the explicitly nested factor.  The recipe is
\end{itemize}
\begin{Schunk}
\begin{Sinput}
> Pastes <- within(Pastes, sample <- factor(batch:cask))
\end{Sinput}
\end{Schunk}
\end{frame}

\begin{frame}
  \frametitle{Avoid implicitly nested representations}
  \begin{itemize}
  \item The \code{lme4} package allows for very general model
    specifications.  It does not require that factors associated with
    random effects be hierarchical or ``multilevel'' factors in the
    design.
  \item The same model specification can be used for data with nested
    or crossed or partially crossed factors.  Nesting or crossing is
    determined from the structure of the factors in the data, not the
    model specification.
  \item You can avoid confusion about nested and crossed factors by
    following one simple rule: ensure that different levels of a
    factor in the experiment correspond to different labels of the
    factor in the data.
  \item Samples were drawn from 30, not 3, distinct casks in this
    experiment.  We should specify models using the \code{sample}
    factor with 30 levels, not the \code{cask} factor with 3 levels.
  \end{itemize}
\end{frame}

\begin{frame}[fragile]
  \frametitle{Pastes data plot}
\includegraphics{figs/data-Pastesplot}
\end{frame}

\begin{frame}[fragile]
  \frametitle{A model with nested random effects}
\begin{Schunk}
\begin{Sinput}
> (fm3 <- lmer(strength ~ 1 + (1 | batch) + (1 | sample), 
+     Pastes))
\end{Sinput}
\begin{Soutput}
Linear mixed model fit by REML 
Formula: strength ~ 1 + (1 | batch) + (1 | sample) 
   Data: Pastes 
 AIC   BIC logLik deviance REMLdev
 255 263.4 -123.5    248.0     247
Random effects:
 Groups   Name        Variance Std.Dev.
 sample   (Intercept) 8.4337   2.9041  
 batch    (Intercept) 1.6573   1.2874  
 Residual             0.6780   0.8234  
Number of obs: 60, groups: sample, 30; batch, 10
Fixed effects:
            Estimate Std. Error t value
(Intercept)  60.0533     0.6768   88.73
\end{Soutput}
\end{Schunk}
\end{frame}

\begin{frame}[fragile]
  \frametitle{Random effects from model fm3}
  \begin{center}
\includegraphics{figs/data-fm3ranef}
  \end{center}
Batch-to-batch variability is low compared to sample-to-sample.
\end{frame}

\begin{frame}[fragile]
  \frametitle{Dimensions and relationships in fm3}
  \begin{itemize}
  \item There are $n=60$ observations, $p=1$ fixed-effects parameter,
    $k=2$ simple, scalar random-effects terms ($q_1=q_2=1$) with
    grouping factors having $n_1=30$ and $n_2=10$ levels.
  \item Because both random-effects terms are simple, scalar terms,
    $\bm\Sigma(\bm\theta)$ is block-diagonal in two diagonal blocks of
    sizes $30$ and $10$, respectively. $\bm Z$ is generated from two
    sets of indicators.
  \end{itemize}
  \begin{center}
\includegraphics{figs/data-Z3fig}
  \end{center}
\end{frame}

\begin{frame}
\frametitle{Eliminate the random-effects term for batch?}
\begin{itemize}
\item We have seen that there is little batch-to-batch variability
  beyond that induced by the variability of samples within batches.
\item We can fit a reduced model without that term and compare it to
  the original model.
\item Somewhat confusingly, model comparisons from likelihood ratio
  tests are obtained by calling the \code{anova} function on the two
  models.  (Put the simpler model first in the call to \code{anova}.)
\item Sometimes likelihood ratio tests can be evaluated using the REML
  criterion and sometimes they can't.  Instead of learning the rules
  of when you can and when you can't, it is easiest always to refit the
  models with \code{REML = FALSE} before comparing.
\end{itemize}
\end{frame}

\begin{frame}[fragile]
  \frametitle{Comparing ML fits of the full and reduced models}
\begin{Schunk}
\begin{Sinput}
> fm3M <- update(fm3, REML = FALSE)
> fm4M <- lmer(strength ~ 1 + (1 | sample), Pastes, REML = FALSE)
> anova(fm4M, fm3M)
\end{Sinput}
\begin{Soutput}
Data: Pastes
Models:
fm4M: strength ~ 1 + (1 | sample)
fm3M: strength ~ 1 + (1 | batch) + (1 | sample)
     Df     AIC     BIC  logLik  Chisq Chi Df Pr(>Chisq)
fm4M  3  254.40  260.69 -124.20                         
fm3M  4  255.99  264.37 -124.00 0.4072      1     0.5234
\end{Soutput}
\end{Schunk}
\end{frame}

\begin{frame}
\frametitle{p-values of LR tests on variance components}
\begin{itemize}
\item The likelihood ratio is a reasonable criterion for comparing
  these two models.  However, the theory behind using a $\chi^2$
  distribution with 1 degree of freedom as a reference distribution
  for this test statistic does not apply in this case.  The null
  hypothesis is on the boundary of the parameter space.
\item Even at the best of times, the p-values for such tests are only
  approximate because they are based on the asymptotic behavior of the
  test statistic.  To carry the argument further, all results in
  statistics are based on models and, as George Box famously said,
  ``All models are wrong; some models are useful.''
\end{itemize}
\end{frame}
\begin{frame}
\frametitle{LR tests on variance components (cont'd)}
\begin{itemize}
\item In this case the problem with the boundary condition results in
  a p-value that is larger than it would be if, say, you compared this
  likelihood ratio to values obtained for data simulated from the null
  hypothesis model.  We say these results are ``conservative''.
\item As a rule of thumb, the p-value for the $\chi^2$ test on a
  simple, scalar term is roughly twice as large as it should be.
\item In this case, dividing the p-value in half would not affect our
  conclusion. 
\end{itemize}
\end{frame}

\begin{frame}[fragile]
  \frametitle{Updated model, REML estimates}
\begin{Schunk}
\begin{Sinput}
> (fm4 <- update(fm4M, REML = TRUE))
\end{Sinput}
\begin{Soutput}
Linear mixed model fit by REML 
Formula: strength ~ 1 + (1 | sample) 
   Data: Pastes 
   AIC   BIC logLik deviance REMLdev
 253.6 259.9 -123.8    248.4   247.6
Random effects:
 Groups   Name        Variance Std.Dev.
 sample   (Intercept) 9.9767   3.1586  
 Residual             0.6780   0.8234  
Number of obs: 60, groups: sample, 30
Fixed effects:
            Estimate Std. Error t value
(Intercept)  60.0533     0.5864   102.4
\end{Soutput}
\end{Schunk}
\end{frame}

\begin{frame}
  \frametitle{Recap of the analysis of the Pastes data}
  \begin{itemize}
  \item The data consist of $n=60$ observations on $q_1=30$ samples
    nested within $q_2=10$ batches.  
  \item The data are labelled with a \code{cask} factor with $3$
    levels but that is an implicitly nested factor.  Create the
    explicit factor \code{sample} and ignore \code{cask} from then
    on.
  \item Specification of a model for nested factors is exactly the
    same as specification of a model with crossed or partially crossed
    factors --- provided that you avoid using implicitly nested factors.
  \item In this case the \code{batch} factor was inert --- it did not
    ``explain'' substantial variability in addition to that attributed
    to the \code{sample} factor. We therefore prefer the simpler model.
  \item At the risk of ``beating a dead horse'', notice that, if we had
    used the \code{cask} factor in some way, we would still need to
    create a factor like \code{sample} to be able to reduce the
    model.  The \code{cask} factor is only meaningful within \code{batch}.
  \end{itemize}
\end{frame}

\begin{frame}
  \frametitle{This is all very nice, but $\dots$}
  \begin{itemize}
  \item These methods are interesting but the results are not really
    new. Similar results are quoted in \Emph{Statistical Methods in
      Research and Production}, which is a very old book.
  \item The approach described in that book is actually quite
    sophisticated, especially when you consider that the methods
    described there, based on observed and expected mean squares, are
    for hand calculation --- in pre-calculator days!
  \item Why go to all the trouble of working with sparse matrices and
    all that if you could get the same results with paper and pencil?
    The one-word answer is \Emph{balance}. 
  \item Those methods depend on the data being balanced. The design
    must be completely balanced and the resulting data must also be
    completely balanced.
  \item Balance is fragile.  Even if the design is balanced, a single
    missing or questionable observation destroys the balance.
    Observational studies (as opposed to, say, laboratory experiments)
    cannot be expected to yield balanced data sets.
  \item Also, the models involve only simple, scalar random effects
    and do not incorporate covariates.
  \end{itemize}
\end{frame}

\section[Fixed-effects]{Incorporating fixed-effects terms: classroom}

\begin{frame}[fragile]
  \frametitle{Structure of the classroom data}
  \begin{itemize}
  \item The \code{classroom} data are a cross-section of students
    within classes within schools.  The \code{mathgain} variable is
    the difference in mathematics achievement scores in grade 1 and
    kindergarten.
  \item These data are quite unbalanced.  The distribution of the
    number of students observed per classroom is
\begin{Schunk}
\begin{Sinput}
> xtabs(~xtabs(~classid, classroom))
\end{Sinput}
\begin{Soutput}
xtabs(~classid, classroom)
 1  2  3  4  5  6  7  8  9 10 
42 53 53 61 39 31 14 13  4  2 
\end{Soutput}
\end{Schunk}
\item Similarly, the distribution of the number of classes observed
  per school is
\begin{Schunk}
\begin{Sinput}
> table(xtabs(~schoolid, unique(subset(classroom, select = c(classid, 
+     schoolid)))))
\end{Sinput}
\begin{Soutput}
 1  2  3  4  5  9 
13 34 26 21 12  1 
\end{Soutput}
\end{Schunk}
  \end{itemize}
\end{frame}

\begin{frame}[fragile]
  \frametitle{Twelve schools, each with 5 classrooms}
\includegraphics{figs/data-Schoolsplot}
\end{frame}

\begin{frame}[fragile]
  \frametitle{Simple, ``unconditional'' model for the classroom data}
\begin{Schunk}
\begin{Sinput}
> (fm5 <- lmer(mathgain ~ 1 + (1 | classid) + (1 | schoolid), 
+     classroom))
\end{Sinput}
\begin{Soutput}
Linear mixed model fit by REML 
Formula: mathgain ~ 1 + (1 | classid) + (1 | schoolid) 
   Data: classroom 
   AIC   BIC logLik deviance REMLdev
 11777 11797  -5884    11771   11769
Random effects:
 Groups   Name        Variance Std.Dev.
 classid  (Intercept)   99.228  9.9613 
 schoolid (Intercept)   77.492  8.8030 
 Residual             1028.234 32.0661 
Number of obs: 1190, groups: classid, 312; schoolid, 107
Fixed effects:
            Estimate Std. Error t value
(Intercept)   57.427      1.443   39.79
\end{Soutput}
\end{Schunk}
\end{frame}

\begin{frame}
  \frametitle{Some comments on the ``unconditional'' model}
  \begin{itemize}
  \item In the multilevel modeling literature a model such as
    \code{fm5} that does not incorporate fixed-effects terms for
    demographic characteristics of the student, class or school, is
    called an ``unconditional'' model.
  \item Notice that the dominant level of variability is the residual
    variability.  It is unlikely that random effects for both classes
    and schools are needed when modeling these data.
  \item We have seen in Exercises 2 that there seem to be trends with
    respect to the \code{minority} factor and the \code{mathkind}
    score but no overall trends with respect to \code{sex}.
  \item A coefficient for a continuous covariate, such as
    \code{mathkind}, or for fixed, reproducible levels of a factor
    like \code{sex} or \code{minority} is incorporated in the
    fixed-effects terms.
  \end{itemize}
\end{frame}

\begin{frame}
  \frametitle{Model-building approach}
  \begin{itemize}
  \item Note that these unbalanced data have, for the most part,
    very few classes per school (sometimes as few as 1) and very few
    students per class (also sometimes as few as 1).  Under these
    circumstances, it is optimistic to expect to be able to partition
    the variability across students, classes and schools.
  \item We should consider adding fixed-effects terms and perhaps
    removing one of the random-effects terms.
  \item We will start by incorporating fixed-effects terms then
    revisit the need for both random-effects terms.
  \item We will begin with the fixed-effects terms adopted as a final
    model in chapter 4 of West, Welch and Ga\l{}ecki (2007).
  \item For brevity, we only display the output of model fits as this
    contains enough information to reconstruct the call to \code{lmer}.
  \end{itemize}
\end{frame}

\begin{frame}[fragile]
  \frametitle{A model with fixed-effects terms}
\begin{Schunk}
\begin{Soutput}
Linear mixed model fit by REML 
Formula: mathgain ~ 1 + mathkind + minority + sex + ses + housepov + (1 |      classid) + (1 | schoolid) 
   Data: classroom 
   AIC   BIC logLik deviance REMLdev
 11396 11442  -5689    11390   11378
Random effects:
 Groups   Name        Variance Std.Dev.
 classid  (Intercept)  81.555   9.0308 
 schoolid (Intercept)  77.761   8.8182 
 Residual             734.420  27.1002 
Number of obs: 1190, groups: classid, 312; schoolid, 107
Fixed effects:
             Estimate Std. Error t value
(Intercept) 285.05708   11.02067  25.866
mathkind     -0.47086    0.02228 -21.133
minorityY    -7.75581    2.38494  -3.252
sexF         -1.23457    1.65743  -0.745
ses           5.23967    1.24496   4.209
housepov    -11.43837    9.93674  -1.151
\end{Soutput}
\end{Schunk}
\end{frame}

\begin{frame}
  \frametitle{Where are the p-values?!!}
  \begin{itemize}
  \item The first thing that most users notice is that there are no
    p-values for the fixed-effects coefficients!  Calculating a p-value
    for $H_0:\beta_j=0$ versus $H_a:\beta_j\ne0$ is not as
    straightforward as it may seem.  The ratio called a ``t value'' in
    the output does not have a Student's T distribution under the null
    hypothesis.
  \item For simple models fit to small, balanced data sets one can
    calculate a p-value.  Not so for unbalanced data.  When the number
    of groups and observations are large, approximations don't matter
    --- you can consider the ratio as having a standard normal
    distribution.
  \item The only time that you can calculate an ``exact'' p-value
    and the difference between this and the standard normal dist'n is
    important is for small, balanced data sets, which are
    exactly the cases that appear in text books.  People get very,
    very upset if the values calculated by the software don't agree
    perfectly with the text book answers.
  \item Here, just say a coefficient is ``significant'' if $|t|> 2$.
  \end{itemize}
\end{frame}

\begin{frame}[fragile]
  \frametitle{Removing the insignificant term for sex}
\begin{Schunk}
\begin{Soutput}
Linear mixed model fit by REML 
Formula: mathgain ~ 1 + mathkind + minority + ses + housepov + (1 | classid) +      (1 | schoolid) 
   Data: classroom 
   AIC   BIC logLik deviance REMLdev
 11397 11438  -5691    11390   11381
Random effects:
 Groups   Name        Variance Std.Dev.
 classid  (Intercept)  81.095   9.0053 
 schoolid (Intercept)  77.604   8.8093 
 Residual             734.457  27.1009 
Number of obs: 1190, groups: classid, 312; schoolid, 107
Fixed effects:
             Estimate Std. Error t value
(Intercept) 284.70224   11.00854  25.862
mathkind     -0.47137    0.02227 -21.170
minorityY    -7.78040    2.38379  -3.264
ses           5.25695    1.24455   4.224
housepov    -11.50122    9.92738  -1.159
\end{Soutput}
\end{Schunk}
\end{frame}

\begin{frame}[fragile]
  \frametitle{Removing the insignificant term for housepov}
\begin{Schunk}
\begin{Soutput}
Linear mixed model fit by REML 
Formula: mathgain ~ mathkind + minority + ses + (1 | classid) + (1 | schoolid) 
   Data: classroom 
   AIC   BIC logLik deviance REMLdev
 11403 11439  -5695    11392   11389
Random effects:
 Groups   Name        Variance Std.Dev.
 classid  (Intercept)  82.839   9.1016 
 schoolid (Intercept)  75.036   8.6623 
 Residual             734.608  27.1036 
Number of obs: 1190, groups: classid, 312; schoolid, 107
Fixed effects:
             Estimate Std. Error t value
(Intercept) 282.41821   10.84049  26.052
mathkind     -0.47031    0.02225 -21.137
minorityY    -8.29077    2.33879  -3.545
ses           5.36456    1.24066   4.324
\end{Soutput}
\end{Schunk}
\end{frame}

\begin{frame}
  \frametitle{Prediction intervals on random effects for class}
  \begin{center}
\includegraphics{figs/data-Classpredi}
  \end{center}
\end{frame}

\begin{frame}[fragile]
  \frametitle{Normal probability plot of random effects for class}
  With many levels of the grouping factor, use a normal probability
  plot of the prediction intervals for the random effects.
\begin{Schunk}
\begin{Sinput}
> qqmath(ranef(fm8, post = TRUE))$classid
\end{Sinput}
\end{Schunk}
  \begin{center}
\includegraphics{figs/data-Classpred2}
  \end{center}
\end{frame}

\begin{frame}[fragile]
  \frametitle{Normal probability plot of random effects for school}
  \begin{center}
\includegraphics{figs/data-Schoolpred}
  \end{center}
\end{frame}

\begin{frame}[fragile]
  \frametitle{Refit without random effects for class}
\begin{Schunk}
\begin{Soutput}
Linear mixed model fit by maximum likelihood 
Formula: mathgain ~ mathkind + minority + ses + (1 | schoolid) 
   Data: classroom 
   AIC   BIC logLik deviance REMLdev
 11415 11446  -5702    11403   11401
Random effects:
 Groups   Name        Variance Std.Dev.
 schoolid (Intercept)  97.87    9.893  
 Residual             789.14   28.092  
Number of obs: 1190, groups: schoolid, 107
Fixed effects:
             Estimate Std. Error t value
(Intercept) 282.30277   10.90127  25.896
mathkind     -0.47045    0.02237 -21.027
minorityY    -7.79570    2.35029  -3.317
ses           5.51947    1.24920   4.418
\end{Soutput}
\end{Schunk}
\end{frame}

\begin{frame}[fragile]
  \frametitle{Check if random effects for class are significant}
\begin{Schunk}
\begin{Sinput}
> fm8M <- update(fm8, REML = FALSE)
> anova(fm9M, fm8M)
\end{Sinput}
\begin{Soutput}
Data: classroom
Models:
fm9M: mathgain ~ mathkind + minority + ses + (1 | schoolid)
fm8M: mathgain ~ mathkind + minority + ses + (1 | classid) + (1 | schoolid)
     Df     AIC     BIC  logLik  Chisq Chi Df Pr(>Chisq)
fm9M  6 11415.5 11446.0 -5701.7                         
fm8M  7 11405.5 11441.1 -5695.8 11.967      1  0.0005415
\end{Soutput}
\end{Schunk}
\begin{itemize}
\item Contrary to what we saw in the plots, the random-effects term
  for \code{classid} is significant even in the presence of the
  \code{schoolid} term
\item Part of the reason for this inconsistency is our incorporating
  312 random effects at a ``cost'' of 1 parameter.  In some way we are
  undercounting the number of degrees of freedom added to the model
  with this term.
\end{itemize}
\end{frame}


\begin{frame}[fragile]\frametitle{A large observational data set}
  \begin{itemize}
  \item A large U.S. university (not mine) provided data on the grade
    point score (\code{gr.pt}) by student (\code{id}), instructor
    (\code{instr}) and department (\code{dept}) from a 10 year period.
    I regret that I cannot make these data available to others.
  \item These factors are unbalanced and partially crossed.
  \end{itemize}
\begin{Schunk}
\begin{Sinput}
> str(anon.grades.df)
\end{Sinput}
\begin{Soutput}
'data.frame':   1721024 obs. of  9 variables:
 $ instr   : Factor w/ 7964 levels "10000","10001",..: 1 1 1 1 1 1 1 1 1 1 ...
 $ dept    : Factor w/ 106 levels "AERO","AFAM",..: 43 43 43 43 43 43 43 43 43 43 ...
 $ id      : Factor w/ 54711 levels "900000001","900000002",..: 12152 1405 23882 18875 18294 20922 4150 13540 5499 6425 ...
 $ nclass  : num  40 29 33 13 47 49 37 14 21 20 ...
 $ vgpa    : num  NA NA NA NA NA NA NA NA NA NA ...
 $ rawai   : num   2.88 -1.15 -0.08 -1.94  3.00 ...
 $ gr.pt   : num  4 1.7 2 0 3.7 1.7 2 4 2 2.7 ...
 $ section : Factor w/ 70366 levels "19959 AERO011A001",..: 18417 18417 18417 18417 9428 18417 18417 9428 9428 9428 ...
 $ semester: num  19989 19989 19989 19989 19972 ...
\end{Soutput}
\end{Schunk}%$
\end{frame}

\begin{frame}[fragile]\frametitle{A preliminary model}
\begin{Schunk}
\begin{Soutput}
Linear mixed model fit by REML 
Formula: gr.pt ~ (1 | id) + (1 | instr) + (1 | dept) 
   Data: anon.grades.df 
     AIC     BIC   logLik deviance REMLdev
 3447389 3447451 -1723690  3447374 3447379
Random effects:
 Groups   Name        Variance Std.Dev.
 id       (Intercept) 0.3085   0.555   
 instr    (Intercept) 0.0795   0.282   
 dept     (Intercept) 0.0909   0.301   
 Residual             0.4037   0.635   
Number of obs: 1685394, groups: id, 54711; instr, 7915; dept, 102

Fixed effects:
            Estimate Std. Error t value
(Intercept)   3.1996     0.0314     102
\end{Soutput}
\end{Schunk}
\end{frame}

\begin{frame}
  \frametitle{Comments on the model fit}
  \begin{itemize}
  \item $n=1685394$, $p=1$, $k=3$, $n_1=54711$, $n_2=7915$, $n_3=102$,
    $q_1=q_2=q_3=1$, $q=62728$
  \item This model is sometimes called the ``unconditional'' model in
    that it does not incorporate covariates beyond the grouping factors.
  \item It takes less than an hour to fit an "unconditional" model
    with random effects for student (\code{id}), instructor
    (\code{inst}) and department (\code{dept}) to these data.
  \item Naturally, this is just the first step.  We want to look at
    possible time trends and the possible influences of the
    covariates.
  \item This is an example of what ``large'' and ``unbalanced'' mean
    today.  The size of the data sets and the complexity of the models
    in mixed modeling can be formidable.
  \end{itemize}
\end{frame}

\begin{frame}
  \frametitle{Recap of simple, scalar random-effects terms}
  \begin{itemize}
  \item For \code{lmer} a simple, scalar random effects term is of the
    form \code{(1|F)}.
  \item The number of random effects generated by the $i$th such
    term is the number of levels, $n_i$, of \code{F} (after dropping
    ``unused'' levels --- those that do not occur in the data.  The idea
    of having such levels is not as peculiar as it may seem if, say,
    you are fitting a model to a subset of the original data.)
  \item Such a term contributes $n_i$ columns to $\bm Z$.  These
    columns are the indicator columns of the grouping factor.
  \item Such a term contributes a diagonal block $\sigma^2_i\bm
    I_{n_i}$ to $\bm\Sigma$.  The multipliers $\sigma_i^2$ can be
    different for different terms. The term contributes exactly one
    element (which happens to be $\sigma_i/\sigma$) to $\bm\theta$.
  \end{itemize}
\end{frame}
